{\Large \color{Naranja}Requerimientos del sistema}

\begin{itemize}[label=\textcolor{Lotus}{$\triangleright$}]
    \item {
        {\large \color{Purpura}Requerimientos funcionales}

        % Detallar las funcionalidades principales del sistema.
        % Incluir casos de uso esenciales y casos propuestos por el equipo.

        %%%%%%%%%%%%%%%%%%%%%%%%%
        % Aqui empieza a escribir

        % Mariana
        \textbf{RF1: Autenticación de Usuarios}
        \begin{itemize}[label=\textcolor{Amarillo}{$\triangleright$}]
        \item La aplicación permite el acceso de usuarios mediante credenciales únicas: usuario y contraseña.
        \item Hay autentificación para acceder como usuario (ciudadano) o administrador. 
        \end{itemize} 
        \textbf{RF2: Gestión de Usuarios}
        \begin{itemize}[label=\textcolor{Amarillo}{$\triangleright$}]
        \item Los usuarios podrán crear, editar y eliminar su perfil. 
        \item Los administradores podrán eliminar las cuentas de los usuarios y ver sus datos. A su vez estos tendrán su propia cuenta. 
        \end{itemize}
        \textbf{RF3: Visualización de Incidentes}
        \begin{itemize}[label=\textcolor{Amarillo}{$\triangleright$}] % EN REVISION 2 ESTADOS VISIBLES, RESUELTO VISIBLE MOMENTANEAMENTE
        \item Los incidentes tendrán 3 estados: reportado, en revisión y resueltos 
        \item Los incidentes se verán en el mapa interactivo según su ubicación, estos tendrán un icono correspondiente a su tipo. 
        \item Los usuarios pueden hacer clic en el ícono de un incidente y ver detalles sobre este.
        \end{itemize}
        
        \textbf{RF4: Gestión de Incidentes}
        \begin{itemize}[label=\textcolor{Amarillo}{$\triangleright$}]
        \item El sistema debe permitir a los usuarios registrar un incidente, para esto se pedirá la siguiente información obligatoriamente:
        \begin{itemize} [label=\textcolor{Amarillo}{$\triangleright$}]
            \item Tipo del incidente, esta será una selección de los tipos predeterminados: baches, luminarias descompuestas, obstáculos en la vía pública, accidentes automovilísticos, otro.
            \item Descripción del incidente.
            \item Fotos de prueba del incidente.
            \item Ubicación, esta se podrá registrar manualmente (de forma escrita o seleccionando el punto en el mapa), o utilizando la ubicación en tiempo actual del usuario. 
        \end{itemize}
        \item Usuarios que mande pruebas de dicho incidente esté resuelto, el administrador podrá decidir, con ese o más reportes si cambiar el estado del incidente como resuelto.
        \item Los incidentes se borrarán después de cierto tiempo cuando ya estén en el estado resuelto.
        \item Los administradores deben recibir notificaciones de nuevos incidentes registrados.
        \end{itemize}

        \textbf{RF5: Filtro de incidentes}
        \begin{itemize}[label=\textcolor{Amarillo}{$\triangleright$}]
        \item Los usuarios podrán filtrar los incidentes en el mapa por su estado (Reportado, en revisión, resuelto) o por su tipo, fecha o estado. 
        \end{itemize}
    }
    
    \item {
        {\large \color{Purpura}Requerimientos no funcionales}

        % Seguridad, rendimiento, disponibilidad, usabilidad, entre otros (pueden considerar requerimientos de dominio).
        
        %%%%%%%%%%%%%%%%%%%%%%%%%
        % Aqui empieza a escribir
        \textbf{Seguridad:}\\
        Definición de protocolo http, acceso restringido a la base de datos, protección de la rama main, los usuarios que reporten incidentes deben estar obligatoriamente registrados en la aplicación, solo administrador puede eliminar y cambiar el estado de cualquier incidente. 

        \textbf{Usabilidad:}\\
        Regulación de contenido, interfaz intuitiva y amigable con el usuario. 
        
        \textbf{Rendimineto:}\\
        El sistema deberá soportar al menos 70 usuarios.\\
        Los datos de incidentes son actualizados en tiempo real.\\
        El procesado y subida de incidentes tarda 5 segundos en promedio.\\
        Filtrar incidentes por tipo o tiempo demora aproximadamente de 10 segundos.\\
        Encontrar incidentes cercanos tarda alrededor de de 20 segundos.

        \textbf{Disponibilidad:}\\
        El sistema deberá estar disponible el 99\% del tiempo.
    }
\end{itemize}