{\Large \color{Naranja}Casos de uso}

% Incluir al menos cuatro casos de uso detallados.
% Presentar los flujos principales y alternativos.

%%%%%%%%%%%%%%%%%%%%%%%%%
% Aqui empieza a escribir


%Emma

\begin{itemize}[label=\textcolor{Lotus}{$\triangleright$}]
    \item {
        \textbf{\color{Purpura}CU1: Reporte de Accidentes:}

        \textbf{\textit{Actores:} Ciudadano}

        \textbf{Flujo Principal:}
    
        \begin{itemize}[label=\textcolor{Amarillo}{$\triangleright$}]
            \item Abre la aplicación web.
            \item Le damos click en el botón de ``$+$'' para poder agregar el accidente.
            \item Llenamos la información requerida sobre el accidente.
            \item Se deberá de indicar correctamente la dirección del accidente.
            \item Cuando se termine de llenar la información, se deberá de presionar agregar el accidente.
            \item El accidente deberá de aparecer en el mapa.
        \end{itemize}
    }
    
    \item {
        \textbf{\color{Purpura}CU2: Gestión y Seguimiento de Reportes:}

        \textbf{\textit{Actores:} Administradores y  Autoridades}

        \textbf{Flujo Principal:}

        \begin{itemize}[label=\textcolor{Amarillo}{$\triangleright$}]
            \item Un administrador entra a la aplicación.
            \item Debe de acceder a la sección correspondiente de los reportes.
            \item Filtra los reportes por grado de importancia y tiempo de publicación.
            \item Puede actualizar el status del reporte: \textit{en trámite, en proceso y solucionado} 
        \end{itemize}
    }
    
    \item {
        \textbf{\color{Purpura}CU3: Estadísticas de Accidentes Automovilísticos:}

        \textbf{\textit{Actores:} Ciudadano, Autoridades y Administradores}

        \textbf{Flujo Principal:}

        \begin{itemize}[label=\textcolor{Amarillo}{$\triangleright$}]
            \item Entrar a la aplicación
            \item Ir al apartado de estadísticas
            \item Poder encontrar una gráfica la cuál pueda describir cuantos accidente ha habido en la semana, mes y año.
            \item Salir de la aplicación.
        \end{itemize}
    }
        
    \item {
        \textbf{\color{Purpura}CU4:Notificación de Accidentes cercanos:}
        
 % en revisión
        \textbf{\textit{Actores:} Ciudadano }

        \textbf{Flujo Principal:}

        \begin{itemize}[label=\textcolor{Amarillo}{$\triangleright$}]
            \item Cada que ocurra una accidente en una vialidad o calle cerca de la ubicación del usuario ó en avenidas importantes, se mandará una notificación informando al usuario para que tome sus precauciones.
        \end{itemize}
    }
\end{itemize}