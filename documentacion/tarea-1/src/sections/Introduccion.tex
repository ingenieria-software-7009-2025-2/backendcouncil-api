{\Large \color{Naranja}Introducción}

%Definir el propósito del sistema.
%Describir brevemente el alcance.
%Incluir definiciones importantes y abreviaciones relevantes.

%%%%%%%%%%%%%%%%%%%%%%%%%
% Aqui empieza a escribir
\begin{itemize}[label=\textcolor{Lotus}{$\triangleright$}]
    \item {\large \color{Purpura}Propósito}
    \\
    Este documento especifica los requerimientos funcionales y no funcionales del Sistema de Reporte de Incidentes Urbanos (o abreviado: SisRep), que permitirá a cualquier usuario registrado en el sistema reportar incidentes de carácter ciudadano, tales como; el mal funcionamiento de alumbrado público, daños a la vía pública (por ejemplo baches), y problemas similares.
    \item {\large \color{Purpura} Alcance}
    \\
    El Sistema de Reporte de Incidentes Urbanos le dará a los ciudadanos una forma fácil y rápida de reportar incidentes que posiblemente aún no sean detectados por las autoridades locales, lo que permitirá que las autoridades correspondientes se hagan cargo de atender y solucionar los reportes de forma mas rápida.  
    \item  {\large \color{Purpura} Definiciones}
    \begin{itemize}[label=\textcolor{Amarillo}{$\triangleright$}]
        \item SisRep: Sistema de Reporte de Incidentes Urbanos .
        \item Usuario: cualquier ciudadano que este registrado en el SisRep.
        \item Reporte: registro de cualquier incidente que cumpla con los requisitos de registro colocado en el SisRep.
    \end{itemize}
\end{itemize}


% Ale